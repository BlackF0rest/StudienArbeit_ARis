%!TEX root = ../main.tex

\chapter{Gegenstand, Ziel und Vorgehensweise}

\section{Ausgangslage und Relevanz der Arbeit}
\ac{AR} gewinnt immer mehr an Bedeutung im industriellen wie auch im privaten Umfeld. Dabei bezeichnet \ac{AR} die Erweiterung der realen Umgebung durch digitale Inhalte wie Texte, Bilder oder 3D-Objekte\autocite[Vgl. S.28 f.]{thomasSmartGlassesAugmented2020}. \ac{AR} ermöglicht somit eine Überlagerung realer und virtueller Objekte in Echtzeit und schafft dadurch neue Formen der Interaktion zwischen Mensch und Maschine\autocite[Vgl. S.28 f.]{thomasSmartGlassesAugmented2020}.\newline
Die zunehmende Relevanz der Thematik spiegelt sich klar in den stetig wachsenden Forschungs- und Wirtschaftstrends \ac{AR} wieder: zwischen 1975 und 2019 stieg die Zahl wissenschaftlicher Publikationen von 229 auf über 120.000\autocite[Vgl. S.20]{thomasSmartGlassesAugmented2020}. Dies zeigt ein klar steigendes wissenschaftliches aber auch gesellschaftliches Interesse an dem gesamten Themenfeld.\newline
Im Kontext der \ac{AR}-Technologien wird in diesem Studienprojekt eine \ac{AR}-Brille auf Basis des Birdbath-Optiksystems entwickelt. Ziel des Projekts ist eine funktionierenden \ac{AR}-Brille zu entwickeln, welche virtuelle Inhalte im Sichtfeld des Benutzers anzeigen kann. Das System soll die Energieversorgung, Elektronik, Optik und Mechanik in einem ergonomischen Gesamtkonzept vereinen.\newline
Des Weiteren soll eine funktionierende Software mit umfangreicher Funktionalität eingeführt werden.
Das Projekt wird in zwei Arbeitsbereiche aufgeteilt. Dabei wird die Elektronik und Mechanik (elektrisches Design, Energieversorgung, Wärmemanagement, Gehäusekonstruktion) von Nura Fasano verantwortet sowie die Optik und Software (Birdbath-Integration, Display-Ansteuerung, Funktionalitäten) von Bastian Kiefer.

\section{Herausforderungen und Zielkonflikte}
Bei dem Entwurf einer Birdbath-AR-Brille bestehen verschiedene Herausforderungen und Zielkonflikte. Zusammenfassen lassen sich diese in konkreten Trade-Offs:

\textbf{Gewicht vs. Batterielaufzeit:} Größere Akkus erhöhen Laufzeit, verschlechtern aber Tragekomfort und Schwerpunktlage\footnote{Lehmann: Industrielle Anwendungen von Augmented Reality, Vgl. S.5.}.\newline
\textbf{Energieeffizienz vs. Bildqualität:} Hohe Helligkeit und Farbtiefe erfordern mehr Leistung; eine Balance zwischen Bildschärfe und Batterieverbrauch ist notwendig\autocite[Vgl. S.1 ff.]{qianPowerConsumptionLight2025}.\newline
\textbf{Optische Qualität vs. Fertigbarkeit:} Präzise Linsen und Spiegel (z. B. Birdbath-Optiken) liefern bessere Abbildung, sind aber schwerer und aufwändiger zu justieren\autocite[Vgl. S.4 ff.]{zhang2024review}.\newline
\textbf{Thermisches Management vs. Designfreiheit:} Lüftungs- oder Kühlstrukturen beeinflussen das äußere Erscheinungsbild und erhöhen Fertigungskomplexität\autocite[Vgl. S.20]{lehmannErmittlungGeeigneterIndustrieller2023}.\newline
\textbf{Kosten vs. Leistung:} Hochwertige Displays und Sensoren steigern Qualität, treiben aber Material- und Entwicklungskosten\autocite[Vgl. S.2 f.]{linAugmentedRealitySmart2025}.

Aus obigen Herausforderungen wird folgende Frage konkretisiert: \textit{,,Wie kann eine kostengünstige, technologisch akutelle aber trotzdem umsetzbare und erweiterbare AR-Brille konzipiert sowie umgesetzt werden?''}

\section{Ziel der Studienarbeit}
Ziel ist die Konstruktion eines Optikkonzepts, welche zusammen mit der Elektronik und Mechanik eine vollintegrierte \ac{AR}-Brille bildet. Dabei soll die Birdbath-Optik in das Gesamtsystem integriert und eine Softwarelösung zur Ansteuerung des Displays und der Sensorik entwickelt werden. Die Software soll verschiedene Funktionalitäten wie die Grundlegende Anzeige von Daten sowie die Einbindung von Sensorendaten und Interaktionsmöglichkeiten bieten.

\section{Vorgehensweise}
Konkret soll somit zuerst ein technisch vollständiges Konzept zur optischen und softwaretechnischen Realisierung einer Birdbath-\ac{AR}-Brille ausgearbeitet werden. Dieses basiert auf einer vollumfänglichen Anforderungsanalyse (\ref{apx:priorisierung_anforderungen}) und umfangreichen technischen Recherchen zu Birdbath-Optiken und Softwarelösungen.

Anschließend soll ein Systemdesign (grob für die Software \ref{apx:software_architektur}) umgesetzt werden, welches die verschiedenen gefundenen Anforderungen und Aspekte in Einklang bringt. Dabei werden verschiedene technische Lösungen hinsichtlich ihrer Umsetzbarkeit, Kosten und Performance bewertet und miteinander verglichen.

Des Weiteren werden verschiedene Prototypen zur Validierung einzelner Komponenten und Teilsysteme entwickelt und getestet. Hierbei liegt der Fokus auf der Integration der Birdbath-Optik in das Gesamtsystem sowie der Ansteuerung des Displays und dem Zusammenspiel der verschiedenen Softwareteile mittels geeigneter Softwarelösungen.

Zuletzt wird die Validierung des Konzepts anhand zu  Beginn definierter Kriterien durchgeführt.
